\documentclass{article}

% packages for math
\usepackage{amsthm}
\usepackage{amsmath}
\usepackage{amssymb}
\usepackage{amsfonts}

% package for including images
\usepackage{graphicx}

% TAKEN FROM OVERLEAF DOCUMENTATION
% https://www.overleaf.com/learn/latex/Code_listing
\usepackage{listings}
\lstset{language=Python}
\usepackage{xcolor}
\definecolor{codegreen}{rgb}{0,0.6,0}
\definecolor{codegray}{rgb}{0.5,0.5,0.5}
\definecolor{codepurple}{rgb}{0.58,0,0.82}
\definecolor{backcolour}{rgb}{0.95,0.95,0.92}
\lstdefinestyle{mystyle}{
    backgroundcolor=\color{backcolour},
    commentstyle=\color{codegreen},
    keywordstyle=\color{magenta},
    numberstyle=\tiny\color{codegray},
    stringstyle=\color{codepurple},
    basicstyle=\ttfamily\footnotesize,
    breakatwhitespace=false,
    breaklines=true,
    captionpos=b,
    keepspaces=true,
    numbers=left,
    numbersep=5pt,
    showspaces=false,
    showstringspaces=false,
    showtabs=false,
    tabsize=2
}
\lstset{style=mystyle}

% environment for solutions
\theoremstyle{remark}
\newtheorem*{solution}{Solution}

% capital letters for problem parts
\renewcommand{\theenumi}{\Alph{enumi}}

% no page numbers
\pagenumbering{gobble}

% UNCOMMENT IF YOU DON'T WANT PROBLEMS ON INDIVIDUAL PAGES
% \renewcommand{\pagebreak}{}

\title{
  Homework 3
}
\author{CAS CS 132: Geometric Algorithms}
\date{Due: \textbf{Thursday September 28, 2023 at 11:59PM}}

\begin{document}
\maketitle

\subsection*{Submission Instructions}
\begin{itemize}
\item Make the answer in your solution to each problem abundantly clear (e.g., put a box around your answer or used a colored font if there is a lot of text which is not part of the answer).
\item Choose the correct pages corresponding to each problem in Gradescope. Note that Gradescope registers your submission as soon as you submit it, so you don't need to rush to choose corresponding pages.
  \textbf{For multipart questions, please make sure each part is accounted for.}
\end{itemize}
Graders have license to dock points if either of the above instructions are not properly followed.


\section*{Practice Problems}

The following list of problems comes from \textit{Linear Algebra and its Application 5th Ed} by David C.\ Lay, Steven R.\ Lay, and Judi J.\ McDonald.
They may be useful for solidifying your understanding of the material and for studying in general.
\textbf{They are optional, so please don't submit anything for them}.

\begin{itemize}
\item (page 40) 1.4.1, 1.4.2
\item (page 41) 1.4.10, 1.4.15, 1.4.16, 1.4.29, 1.4.30
\item (page 61) 1.7.1, 1.7.5
\item (page 62) 1.7.15, 1.7.17, 1.7.35, 1.7.37
\end{itemize}

\pagebreak
\section{Spanning Columns}

(10 points) Do the columns of the following matrix span all of $\mathbb R^3$?
\begin{displaymath}
  \begin{bmatrix}
    1&-5&4\\
    -1&6&-3\\
    -2&13&-7\\
  \end{bmatrix}
\end{displaymath}
Justify your answer. (This means explaining why your answer is correct. You \textbf{do not} have to write down any row operations you use.)

\begin{solution}
\end{solution}

\pagebreak
\section{Matrix-Vector Multiplications}

Write down a matrix $A$ in $\mathbb R^{4 \times 4}$ such that
\begin{displaymath}
  A
  \begin{bmatrix}
    1 \\ 1 \\ 1 \\ 1
  \end{bmatrix}
  =
  \begin{bmatrix}
    1 \\ 2 \\ 3 \\ 4
  \end{bmatrix}
\end{displaymath}
For each part, write down $A$ in the specified shape, where $\blacksquare$ represents a nonzero entry.
{
  \newcommand{\bs}{\blacksquare}
  \begin{enumerate}
  \item (5 points)
    \begin{displaymath}
      \begin{bmatrix}
        \bs&0&0&0 \\
        0&\bs&0&0 \\
        0&0&\bs&0 \\
        0&0&0&\bs
      \end{bmatrix}
    \end{displaymath}
  \item (5 points)
    \begin{displaymath}
      \begin{bmatrix}
        \bs&0&0&0 \\
        \bs&\bs&0&0 \\
        \bs&\bs&\bs&0 \\
        \bs&\bs&\bs&\bs
      \end{bmatrix}
    \end{displaymath}
  \item (5 points)
    \begin{displaymath}
      \begin{bmatrix}
        \bs&\bs&\bs&\bs \\
        \bs&\bs&\bs&0 \\
        \bs&\bs&0&0 \\
        \bs&0&0&0 \\
      \end{bmatrix}
    \end{displaymath}
  \end{enumerate}
}

\begin{solution}
\end{solution}

\pagebreak
\section{Linear Independence}

(10 points) Are the following vectors linearly independent?
\begin{displaymath}
  \mathbf v_1 = \begin{bmatrix}
    1 \\ -1 \\ 2
  \end{bmatrix}
  \qquad
  \mathbf v_2 =
  \begin{bmatrix}
    -1 \\ 4 \\ -3
  \end{bmatrix}
  \qquad
  \mathbf v_3 =
  \begin{bmatrix}
    -3 \\ 9 \\ -8
  \end{bmatrix}
\end{displaymath}
If so, justify your answer. If not, write one of the vectors as a linear combination of the others.
\begin{solution}
\end{solution}

\pagebreak
\section{Linearly Dependent Sets}
(15 points) Write down three \textbf{nonzero} vectors $\mathbf v_1$, $\mathbf v_2$, and $\mathbf v_3$ in $\mathbb R^3$ such that
\begin{itemize}
\item $\{\mathbf v_1, \mathbf v_2, \mathbf v_3\}$ is linearly dependent and
\item $\mathbf v_1$ \textbf{cannot} be written as a linear combination of $\mathbf v_2$ and $\mathbf v_3$.
\end{itemize}
Your solution should be given in the following form:
\begin{displaymath}
  \mathbf v_1 =
  \begin{bmatrix}
    * \\ * \\ *
  \end{bmatrix}
  \qquad
  \mathbf v_2 =
  \begin{bmatrix}
    * \\ * \\ *
  \end{bmatrix}
  \qquad
  \mathbf v_3 =
  \begin{bmatrix}
    * \\ * \\ *
  \end{bmatrix}
\end{displaymath}
where $*$ represents an arbitrary entry.

\begin{solution}
\end{solution}

\pagebreak
\section{Linear Independence, Computationally}

Consider the following matrix, presented as a numpy array.
\begin{lstlisting}
a = np.array(
    [[ 13., -19,  19,  16,   5,   1,  10,   5,  15],
     [-11,   -7, -10,   7,   2,  -8, -10, -19,   6],
     [ -5,   10,  -7,   2,  -8,   2, -15, -16, -11],
     [ 17,  -13,   9,  13,  19,   8,  -3,  -9,   0],
     [  9,  -18,   5,   1,   4,  14,   9,   8,  -4],
     [  8,   14,  17,   5,  -6,   7, -13,   2,  12],
     [ 18,   12,  -7,   2, -10,  15, -12,   1, -12],
     [ 19,  -12,   1, -16,   2,  -6,  -4,  17,  15],
     [-19,   -6, -16, -20, -20,  -3,   7,   3,  14],
     [  6,    8, -15,   5,  -5,   8, -14,   5, -19]])
\end{lstlisting}

\begin{enumerate}
\item (5 points) Do the columns of this matrix span all of $\mathbb R^{10}$? Justify your answer.
\item (5 points) Are the columns of this matrix linearly independent? Justify your answer.
\end{enumerate}
You may use python to solve this problem (I recommend it, you can copy-paste the above lines).
You must describe what you did in python to come to your solution.

\begin{solution}
\end{solution}

\pagebreak
\section{A Small Interface}

(20 points) This week you will be filling in a small interface for answering some of the questions we have considered so far in the course.
\textbf{Read through the docstring of each function carefully.}
You will see what you need to implement described there.

You are given starter code in the file \texttt{hw03prog.py}.
You are also provided with a file \texttt{gauss.py} which implements gaussian elimination.
You are only required to submit \texttt{hw03prog.py}, a copy of \texttt{gauss.py} will be available to the autograder.
\textbf{Don't change the name of this file when you submit.}
Also don't change any of the names of the functions included in the starter code.
\textbf{The only changes you should make are to fill in the TODO items in the starter code}.

Each function can be written in a single line (you are not required to do so).
This is to say that the code is not lengthy, it just requires that you understand what each function is supposed to do.

Some guidelines for this assignment (and others as well):
\begin{itemize}
\item Work incrementally. Don't try to implement the entire program in one go and then debug.
\item Write some test cases, or at least check your implemenetations in the command line.
  We are not providing any tests this week.
\end{itemize}

You will upload the single python file \texttt{hw03prog.py} to Gradescope with your implementations of the required functions.
We will be running autograder tests on your submission to determine its correctness.
\textbf{You will not have access to the autograder tests.}

\end{document}
