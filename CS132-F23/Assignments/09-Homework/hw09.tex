\documentclass{article}

% packages for math
\usepackage{amsthm}
\usepackage{amsmath}
\usepackage{amssymb}
\usepackage{amsfonts}

% package for including images
\usepackage{graphicx}

% TAKEN FROM OVERLEAF DOCUMENTATION
% https://www.overleaf.com/learn/latex/Code_listing
\usepackage{listings}
\lstset{language=Python}
\usepackage{xcolor}
\definecolor{codegreen}{rgb}{0,0.6,0}
\definecolor{codegray}{rgb}{0.5,0.5,0.5}
\definecolor{codepurple}{rgb}{0.58,0,0.82}
\definecolor{backcolour}{rgb}{0.95,0.95,0.92}
\lstdefinestyle{mystyle}{
  backgroundcolor=\color{backcolour},
  commentstyle=\color{codegreen},
  keywordstyle=\color{magenta},
  numberstyle=\tiny\color{codegray},
  stringstyle=\color{codepurple},
  basicstyle=\ttfamily\footnotesize,
  breakatwhitespace=false,
  breaklines=true,
  captionpos=b,
  keepspaces=true,
  numbers=left,
  numbersep=5pt,
  showspaces=false,
  showstringspaces=false,
  showtabs=false,
  tabsize=2
}
\lstset{style=mystyle}

% environment for solutions
\theoremstyle{remark}
\newtheorem*{solution}{Solution}

% capital letters for problem parts
\renewcommand{\theenumi}{\Alph{enumi}}

% no page numbers
\pagenumbering{gobble}

% UNCOMMENT IF YOU DON'T WANT PROBLEMS ON INDIVIDUAL PAGES
% \renewcommand{\pagebreak}{}

\newcommand{\vv}[1]{\mathbf{#1}}
\newcommand{\R}{\mathbb R}
\DeclareMathOperator{\vspan}{span}
\DeclareMathOperator{\cod}{cod}
\DeclareMathOperator{\ran}{ran}
\DeclareMathOperator{\col}{Col}
\DeclareMathOperator{\nul}{Nul}
\DeclareMathOperator{\rank}{rank}

\title{
  Homework 9
}
\author{CAS CS 132: Geometric Algorithms}
\date{Due: \textbf{Thursday November 16, 2023 at 11:59PM}}

\begin{document}
\maketitle

\subsection*{Submission Instructions}
\begin{itemize}
\item Make the answer in your solution to each problem abundantly clear (e.g., put a box around your answer or used a colored font if there is a lot of text which is not part of the answer).
\item Choose the correct pages corresponding to each problem in Gradescope. Note that Gradescope registers your submission as soon as you submit it, so you don't need to rush to choose corresponding pages.
  \textbf{For multipart questions, please make sure each part is accounted for.}
\end{itemize}
Graders have license to dock points if either of the above instructions are not properly followed.


\section*{Practice Problems}

The following list of problems comes from \textit{Linear Algebra and its Application 5th Ed} by David C.\ Lay, Steven R.\ Lay, and Judi J.\ McDonald.
They may be useful for solidifying your understanding of the material and for studying in general.
\textbf{They are optional, so please don't submit anything for them}.

\begin{itemize}
\item 5.1.1, 5.1.3, 5.1.8, 5.1.13, 5.1.23, 5.1.30
\item 5.2.1, 5.2.3, 5.2.7, 5.2.17, 5.2.18
\end{itemize}

\pagebreak
\section{Eigenvalues and Eigenvectors}

Consider the following matrix.
\begin{displaymath}
  A =
  \begin{bmatrix}
    -17 & 28 & 14 \\
    -7 & 11 & 7 \\
    -7 & 14 & 4
  \end{bmatrix}
\end{displaymath}
\begin{enumerate}
\item (5 points)
  Determine if the following vectors are eigenvectors of $A$. For the ones that are, find their associated eigenvalues. Show your work. You may use Python, but if you do, you must include the lines of code you used.
  \begin{displaymath}
    \vv v_1 =
    \begin{bmatrix}
      1 \\ 1 \\ 1
    \end{bmatrix}
    \qquad
    \vv v_2 =
    \begin{bmatrix}
      2 \\ 2 \\ 0
    \end{bmatrix}
    \qquad
    \vv v_3 =
    \begin{bmatrix}
      2 \\ 1 \\ 1
    \end{bmatrix}
  \end{displaymath}
\item (6 points)
  Show that $-3$ is an eigenvalue of $A$ \textit{without doing any row operations}.
  Use the invertible matrix theorem to justify your answer.
\item (6 points)
  Find a basis for the eigenspace of $A$ corresponding to the eigenvalue $3$.
\end{enumerate}

\medskip

\begin{solution}
\end{solution}

\pagebreak
\section{Determinants}

\begin{enumerate}
\item (4 points)
  Compute the determinant of
  \begin{displaymath}
    \begin{bmatrix}
      3 & -3 & 0 \\
      0 & 3 & -1 \\
      2 & 0 & -1
    \end{bmatrix}
  \end{displaymath}
\item (5 points) Given $\det A = 3.5$ and $\det B = -2$, find the determinant of the matrix $B(AB)^{-1}(AB)^TA$.
\item (6 points) Consider the following reduction sequence
  \begin{align*}
    R_1 &\gets R_1 + R_2 \\
    \mathsf{swap}&(R_2, R_3) \\
    R_3 &\gets R_3 + 5R_4 \\
    R_2 &\gets -3R_2 \\
    R_5 &\gets R_5 - 10R_3 \\
    R_5 &\gets R_5 / 11 \\
    \mathsf{swap}&(R_5, R_3) \\
    \mathsf{swap}&(R_1, R_2) \\
    R_4 &\gets R_4 + R_1 \\
    R_2 &\gets 5R_2 \\
    R_1 &\gets -R_1
  \end{align*}
  Suppose that $A \in \R^{5 \times 5}$ reduces to $U$ by this reduction sequence, where $U$ is in \textit{reduced} echelon form.
  If $\rank A = 5$, what is $\det A$?
\item (2 points)
  What is $\det A$ from the previous part if $\rank A = 4$?
\end{enumerate}

% TODO

\medskip

\begin{solution}
\end{solution}

\pagebreak
\section{Properties of Determinants}

For each of the statements, either argue that it is true for any choice of matrices or give counterexamples in $\R^2$ showing it is false.
All matrices are assumed to be square.

\begin{enumerate}
\item (3 points) If $A \sim B$ ($A$ is row equivalent to $B$) then $\det(A) = \det(B)$.
\item (3 points) $\det(5A) = 5\det A$
\item (3 points) $\det(A^TA) \geq 0$
\item (3 points) $\det(A + B) = \det(A) + \det(B)$
\item (3 points) $\det(ABA^{-1}) = \det(B)$
\end{enumerate}

\medskip

\begin{solution}
\end{solution}

\pagebreak
\section{Characteristic Polynomials}

\begin{enumerate}
\item (3 points) Find the characteristic polynomial of
  \begin{displaymath}
    A = \begin{bmatrix}
      1 & -1\\
      -1 & 3 \\
    \end{bmatrix}
  \end{displaymath}
  and factor it in order to determine the eigenvalues $A$.
\item (4 points) Find the characteristic polynomial of
  \begin{displaymath}
    \begin{bmatrix}
      1 & 0 & 0 & 0 & 0 \\
      -1 & 5 & 0 & 0 & 0 \\
      2 & 6 & 3 & 0 & 0 \\
      10 & -15 & 3 & 4 & 0 \\
      -1 & 5 & 2 & 5 & 5
    \end{bmatrix}
  \end{displaymath}
  and use it to determine the eigenvalues of $A$.
\item (5 points) Find the characteristic polynomial of
  \begin{displaymath}
    \begin{bmatrix}
      1 & 0 & 2 & 10 & 5 \\
      0 & 0 & 5 & -3 & 15 \\
      0 & 0 & 16 & 6 & -1 \\
      0 & 0 & 0 & 1 & 5\\
      0 & 0 & 0 & 7 & 4
    \end{bmatrix}
  \end{displaymath}
  You do not have to factor the polynomial, but your expression should not contain any fractions.
\item (6 points) Find the characterstic polynomial of
  \begin{displaymath}
    \begin{bmatrix}
      1 & 0 & 0 \\
      1 & 2 & 5 \\
      0 & 1 & 3
    \end{bmatrix}
  \end{displaymath}
  You do not have to factor the polynomial, but your expression should not contain any fractions.
  \textit{Hint.} Try to row reduce $A - \lambda I$ as usual, scaling rows before zeroing out entries.
  At the end, the scalings you performed will divide the final polynomial.
\end{enumerate}


\medskip

\begin{solution}
\end{solution}

\pagebreak
\section{Closed-Form Solution for Fibonacci}

Consider the matrix
\begin{displaymath}
  A =
  \begin{bmatrix}
    1 & 1 \\
    1 & 0
  \end{bmatrix}
\end{displaymath}

\begin{enumerate}
\item (3 points)
  Verify that
  \begin{displaymath}
    \vv v_1 =
    \begin{bmatrix}
      \frac{1 + \sqrt{5}}{2} \\ 1
    \end{bmatrix}
    \qquad
    \vv v_2 =
    \begin{bmatrix}
      \frac{1 - \sqrt{5}}{2} \\ 1
    \end{bmatrix}
  \end{displaymath}
  form an eigenbasis of $A$ (recall that this means showing they are eigenvectors of $A$, and form a basis of $\R^2$).
  Also determine the eigenvalues for each eigenvector.
\item (3 points)
  Write the vector $[1 \ 0]^T$ in terms of the eigenbasis you found.
  In other words, determine $\alpha_1$ and $\alpha_2$ such that
  \begin{displaymath}
    \begin{bmatrix}
      1 \\ 0
    \end{bmatrix}
    =
    \alpha_1\vv v_1 + \alpha_2 \vv v_2
  \end{displaymath}
  \textit{Hint}. Don't try to do any row reductions, this will be a messy calculation.
  Calcuate $\vv v_1 - \vv v_2$.
\item (4 points) Write down a closed-form solution for the linear dynamical system determined by $A$ with initial vector $[1 \ 0]^T$.
\item (3 points) If you look at the formula given by the second component of your closed-form solution from the previous part, this gives a \textit{non-recursive} definition for Fibonacci numbers.
  Write down this formula and use it (and Python or a calculator) to calculate $F_{20}$, the $20$th fibonacci number (where $F_0 = 0$ and $F_1 = 1$).\footnote{You should verify that this is correct by writing a Python function which computes Fibonacci numbers in the usual way, but you do not have to do this.}
\end{enumerate}

\medskip

\begin{solution}
\end{solution}
\vfill

\end{document}
