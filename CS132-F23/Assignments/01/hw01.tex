\documentclass{article}

% packages for math
\usepackage{amsthm}
\usepackage{amsmath}

% package for including images
\usepackage{graphicx}

% TAKEN FROM OVERLEAF DOCUMENTATION
% https://www.overleaf.com/learn/latex/Code_listing
\usepackage{listings}
\lstset{language=Python}
\usepackage{xcolor}
\definecolor{codegreen}{rgb}{0,0.6,0}
\definecolor{codegray}{rgb}{0.5,0.5,0.5}
\definecolor{codepurple}{rgb}{0.58,0,0.82}
\definecolor{backcolour}{rgb}{0.95,0.95,0.92}
\lstdefinestyle{mystyle}{
    backgroundcolor=\color{backcolour},
    commentstyle=\color{codegreen},
    keywordstyle=\color{magenta},
    numberstyle=\tiny\color{codegray},
    stringstyle=\color{codepurple},
    basicstyle=\ttfamily\footnotesize,
    breakatwhitespace=false,
    breaklines=true,
    captionpos=b,
    keepspaces=true,
    numbers=left,
    numbersep=5pt,
    showspaces=false,
    showstringspaces=false,
    showtabs=false,
    tabsize=2
}
\lstset{style=mystyle}

% environment for solutions
\theoremstyle{remark}
\newtheorem*{solution}{Solution}

% capital letters for problem parts
\renewcommand{\theenumi}{\Alph{enumi}}

% no page numbers
\pagenumbering{gobble}

% UNCOMMENT IF YOU DON'T WANT PROBLEMS ON INDIVIDUAL PAGES
% \renewcommand{\pagebreak}{}

\title{Homework 1}
\author{CAS CS 132: Geometric Algorithms}
\date{Due: \textbf{September 14 at 2PM EST}}

\begin{document}
\maketitle

\subsection*{Submission Instructions}
\begin{itemize}
\item Make the answer in your solution to each problem abundantly clear (e.g., put a box around your answer or used a colored font if there is a lot of text which is not part of the answer).
\item Choose the correct pages corresponding to each problem in Gradescope. Note that Gradescope registers your submission as soon as you submit it, so you don't need to rush to choose corresponding pages.
\end{itemize}
Graders have license to dock points if either of the above instructions are not properly followed.

\pagebreak
\section{Solving Systems of Linear Equations}

Consider the following system of linear equations.
\begin{align*}
  3x - 4y + 3z &= -9 \\
  6x + 7y - 3z &= 0 \\
  x + 10z &= -21
\end{align*}
\begin{enumerate}
\item (2 pts) Write down the augmented matrix for this system.
\item (8 pts)
  Solve the system using elementary row operations.
  Write down the row operations you used as we did in lecture.
\end{enumerate}

\begin{solution}
  % TODO
\end{solution}

\pagebreak
\section{Plane Intersection}

(10 pts) Write the slope-intercept form of the line equation which defines the intersection of the plane
\begin{displaymath}
  2x + 3y + 3z = 6
\end{displaymath}
with the $xy$-plane.

\begin{solution}
  % TODO
\end{solution}

\pagebreak
\section{Homogeneous Systems}

A system of linear equations is \textit{homogeneous} if it is of the form
\begin{align*}
  a_{11}x_1 + a_{12}x_2 + \dots + a_{1n}x_n &= 0\\
  a_{21}x_1 + a_{22}x_2 + \dots + a_{2n}x_n &= 0\\
  &\vdots \\
  a_{m1}x_1 + a_{m2}x_2 + \dots + a_{mn}x_n &= 0\\
\end{align*}
That is, all of its equations have $0$ on the right-hand side.
Consider the following pair of systems
\begin{align*}
  ax + by &= c \\
  dx + ey &= f
\end{align*}
\begin{align*}
  ax + by - cz = 0 \\
  dx + ey - fz = 0
\end{align*}
\begin{enumerate}
\item (5 pts) Show that if the first system has a solution, then so does the second homogeneous one.
\item (5 pts) Give \textbf{nonzero} values to $a$ through $f$ such that the second system has a solution, but the first does not. Present your solution as an augmented matrix for the first system, i.e., of the form
  \begin{displaymath}
    \begin{bmatrix}
      a & b & c \\
      d & e & f
    \end{bmatrix}
  \end{displaymath}
\end{enumerate}

\begin{solution}
  % TODO
\end{solution}

\pagebreak
\section{Composing Systems of Linear Equations}

Consider the following pair of systems of linear equations.
\begin{align*}
  x_1 - 2x_2 &= 3 \\
  4x_1 + x_2 &= 21
\end{align*}
\begin{align*}
  10x_3 + 2x_4 &= x_1 \\
  (-8)x_3 + 9x_4 &= x_2
\end{align*}
\begin{enumerate}
\item (5 pts) Solve the first system of linear equations (in $x_1$ and $x_2$) and write down the augmented matrix of the second system with the solutions of $x_1$ and $x_2$ substituted in.
\item (5 pts) Write the augmented matrix of a \textit{single} system of linear equations in the variables $x_1$, $x_2$, $x_3$, $x_4$ (in that order). Discuss the relationship between this matrix and the one in the previous part. (Hint. Move around $x_1$ and $x_2$ in the second system.)
\end{enumerate}

\begin{solution}
  % TODO
\end{solution}

\pagebreak
\section{Pairwise Consistency}

Given an example of a system of linear equations in 3 variables such that every pair of equations is consistent, but the system as a whole is inconsistent. That is, give values for $a$ through $l$ in the system
\begin{align*}
  ax + by + cz = d \\
  ex + fy + gz = h \\
  ix + jy + kz = l
\end{align*}
such that the system is inconsistent, but each pair of equations forms a consistent system. Present your solution to each part as the an augmented matrix, i.e., of the form
\begin{displaymath}
  \begin{bmatrix}
    a & b & c & d \\
    e & f & g & h \\
    i & j & k & l \\
  \end{bmatrix}
\end{displaymath}
(Hint. Make the third equation inconsistent with the sum of the first and second equation)
\begin{enumerate}
\item (5 pts) Achieve this with no more than 5 nonzero values for $a$ through $l$.
\item (5 pts) Achieve this with all nonzero values for $a$ through $l$.
\end{enumerate}

\begin{solution}
  % TODO
\end{solution}

\pagebreak
\section{More than 2 Solutions}

(10 pts) Consider the system of linear equations given by
\begin{align*}
  ax + by + cz = d \\
  ex + fy + gz = h \\
  ix + jy + kz = l
\end{align*}
where the values $a$ through $l$ are fixed real numbers.
Show that if $(s_x, s_y, s_z)$ and $(t_x, t_y, t_z)$ are solutions to the above system, then
\begin{displaymath}
  \left(\frac{s_x + t_x}{2}, \frac{s_y + t_y}{2}, \frac{s_z + t_z}{2} \right)
\end{displaymath}
is also a solution.
Discuss why this implies the system has infinitely many solutions.

\begin{solution}
  % TODO
\end{solution}

\pagebreak
\section{Lines in 3 Dimensions}

(10 pts) As we discussed in lecture, a 3 dimensional linear equation does not describe a line, but a plane.
One way to describe a line in 3 dimensions is to use a parameter $t$.
That is, after fixing values $a_x$, $b_x$, $a_y$, $b_y$, $a_z$, and $b_z$, a line can be described as all points of the form
\begin{displaymath}
  (a_xt + b_x, a_yt + b_y, a_zt + b_z)
\end{displaymath}
for any real value of $t$ (make sure to convince yourself of this!).
We also saw the intersection of two planes can describe a line.

Give a pair of 3 dimensional linear equations (each of which represents a plane) whose intersection is exactly the line whose points are defined by the above parametric form. (Hint. The line above can be thought of as a system in the variables $x$, $y$, $z$, and $t$, e.g., one of its equations is $x - a_xt = b_x$. Write $t$ in terms of $x$ and substitute this value for $t$ into the other equations.)

\begin{solution}
  % TODO
\end{solution}

\pagebreak
\section{(Programming) Floating-Point Accuracy}

In this problem you will be verifying that you've set everything up Python correctly and that you are comfortable submitting coding assignments on Gradescope.
We'll also be reasoning a bit about floating point error.

\textbf{NOTE. The coding submission on gradescope is just a practice run this week. It's worth 0 points, and this question is graded in the analytic part. But you will NOT receive credit if you do not submit your code via Gradescope as well.}

Consider the function
\begin{lstlisting}
  def sub_error(n):
    return abs(n / 7 / 10 * 7 - n / 10)
\end{lstlisting}
We will be looking at the way this error varies as $x$ increases.
\begin{enumerate}
\item (5 pts) Fill in the function
\begin{lstlisting}
def next_error(start):
  # TODO
  return 0.0
\end{lstlisting}
which returns the returns \texttt{sub\_error(x')} where \texttt{x'} is the first value greater than or equal to \texttt{x} out of \texttt{x, x + 1, x + 2,...} such that \texttt{sub\_error(x')} is nonzero.
\textbf{Please reproduce your function in your submission to the analytic part of the assignment.}

\item (3 pts) Plot \texttt{next\_error} on a log scale graph using \texttt{matplotlib} (you don't have to write any code for this part, its included in the starter code). And include a copy of the image produced in your solution.

\item (2 pts) Discuss the trend in the plot. Give an informal justification for its shape.
\end{enumerate}

\begin{solution}
  % TODO
\end{solution}
\end{document}
