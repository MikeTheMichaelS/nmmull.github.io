\documentclass{article}

% packages for math
\usepackage{amsthm}
\usepackage{amsmath}
\usepackage{amsfonts}

% package for including images
\usepackage{graphicx}

% TAKEN FROM OVERLEAF DOCUMENTATION
% https://www.overleaf.com/learn/latex/Code_listing
\usepackage{listings}
\lstset{language=Python}
\usepackage{xcolor}
\definecolor{codegreen}{rgb}{0,0.6,0}
\definecolor{codegray}{rgb}{0.5,0.5,0.5}
\definecolor{codepurple}{rgb}{0.58,0,0.82}
\definecolor{backcolour}{rgb}{0.95,0.95,0.92}
\lstdefinestyle{mystyle}{
    backgroundcolor=\color{backcolour},
    commentstyle=\color{codegreen},
    keywordstyle=\color{magenta},
    numberstyle=\tiny\color{codegray},
    stringstyle=\color{codepurple},
    basicstyle=\ttfamily\footnotesize,
    breakatwhitespace=false,
    breaklines=true,
    captionpos=b,
    keepspaces=true,
    numbers=left,
    numbersep=5pt,
    showspaces=false,
    showstringspaces=false,
    showtabs=false,
    tabsize=2
}
\lstset{style=mystyle}

% environment for solutions
\theoremstyle{remark}
\newtheorem*{solution}{Solution}

% capital letters for problem parts
\renewcommand{\theenumi}{\Alph{enumi}}

% no page numbers
\pagenumbering{gobble}

% UNCOMMENT IF YOU DON'T WANT PROBLEMS ON INDIVIDUAL PAGES
% \renewcommand{\pagebreak}{}

\title{Homework 2}
\author{CAS CS 132: Geometric Algorithms}
\date{Due: \textbf{September 21 at 2PM EST}}

\begin{document}
\maketitle

\subsection*{Submission Instructions}
\begin{itemize}
\item Make the answer in your solution to each problem abundantly clear (e.g., put a box around your answer or used a colored font if there is a lot of text which is not part of the answer).
\item Choose the correct pages corresponding to each problem in Gradescope. Note that Gradescope registers your submission as soon as you submit it, so you don't need to rush to choose corresponding pages.
\end{itemize}
Graders have license to dock points if either of the above instructions are not properly followed.

\pagebreak
\section{$(2 \times 2)$ Echelon Forms}

(10 pts) Write down all the $2 \times 2$ matrices in echelon form whose entries are either $0$ or $1$.
Mark which ones are in reduced echelon form.

\begin{solution}
\end{solution}

\pagebreak
\section{Converting to Reduced Echelon Form}

(10 pts) Convert the following matrix into reduced echelon form.
Write down which columns are pivot columns.

\begin{displaymath}
  \begin{bmatrix}
    5 & 2 & 9 \\
    7 & 3 & 12 \\
    2 & 1 & 3
  \end{bmatrix}
\end{displaymath}

\begin{solution}
\end{solution}

\pagebreak
\section{Switching Free and Bound Variables}
Consider the following matrix in reduced echelon form.
\begin{displaymath}
  \begin{bmatrix}
    1 & 1 & 0 & 2 & 3 \\
    0 & 0 & 1 & 1 & 4
  \end{bmatrix}
\end{displaymath}
\begin{enumerate}
\item (5 pts) Write down the solution to this system in general form.
\item\label{b} (5 pts) Rewrite the solution in general form so that $x_1$ and $x_3$ are free.
\item\label{c} (5 pts) Write down the solution from part \ref{b} as an augmented matrix (that is, your general form solution from part \ref{b} should have two linear equations, you should rearrange these equations and write them as an augmented matrix).
  Write down the row operation of the form $R_i \gets R_i + cR_j$ which converts the matrix from this part to the one above.
\end{enumerate}

\begin{solution}
\end{solution}

\pagebreak
\section{Linear Combinations}
(10 pts) Is the vector
$
\begin{bmatrix}
  0 \\ -3 \\ -20
\end{bmatrix}
$
in the span of
$
\begin{bmatrix}
  1 \\ 1 \\ 3
\end{bmatrix}
$
and
$
\begin{bmatrix}
  1 \\ 2 \\ 8
\end{bmatrix}?
$
Justify your answer.

\begin{solution}
\end{solution}

\pagebreak
\section{Creating New Spans}
(15 pts) Consider the vectors
\begin{displaymath}
  \mathbf v_1 =
  \begin{bmatrix}
    1 \\ 1 \\ 1 \\ 1
  \end{bmatrix}
  \qquad
  \mathbf v_2 =
  \begin{bmatrix}
    1 \\ 1 \\ 2 \\ 2
  \end{bmatrix}
\end{displaymath}
Write vectors $\mathbf v_3$ and $\mathbf v_4$ such that $\mathbf v_3$ is not in $\mathsf{span}\{\mathbf v_1, \mathbf v_2\}$ and $\mathbf v_4$ is not in $\mathsf{span}\{\mathbf v_1, \mathbf v_2, \mathbf v_3\}$.

\begin{solution}
\end{solution}

\pagebreak
\section{(Programming) Gaussian Elimination}

(20 pt) As you might have guessed, this week you will be implementing Gaussian elimination.
The point of this exercise is to wrestle with the pseudocode and turn it into working Python code.
By the end, you should have a python program which you can use to solve the systems of linear equations we see in this course.

You are given starter code in the file \texttt{hw02prog.py}.
Please do not change the name of this file when you submit.
Also do not change any of the names of the functions included in the starter code.
You may add your own functions, but you are not expected to.
All you are required to do is \textbf{fill in the TODO items in the starter code}.

You will not be expected to handle floating point error perfectly.
This is the purview of numerical analysis, and although it is extremely important in real world applications of linear algebra, it is quite difficult, and not as important as understanding the algorithm.

Some guidelines for this assignment (and others as well):
\begin{itemize}
\item Use the provided functions when possible. In particular, \texttt{zero\_in\_pivot\_column} covers a bit of floating-point error.
\item Take a look at the functions in the numpy library, it's good practice to build of your knowledge of a library as you use it.
  In particular, look at \texttt{np.nonzero} (e.g., look at the online documentation), it will be useful.
\item Work incrementally. Don't try to implement the entire program in one go and then debug.
\end{itemize}
There are two test cases in the starter code.
These are, of course, not exhaustive.
Also note that if you want to do more tests, you can load text files of numpy matrices using \texttt{np.loadtxt} (see the documentation).

You will upload a single python file to Gradescope with your implementations of the required functions.
We will be running autograding tests on your submission to determine its correctness.
\textbf{You will not have access to the autograder tests.}

\end{document}
