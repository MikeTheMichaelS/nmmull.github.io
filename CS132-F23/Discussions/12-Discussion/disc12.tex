\documentclass{article}

% packages for math
\usepackage{amsthm}
\usepackage{amsmath}
\usepackage{amssymb}
\usepackage{amsfonts}

\usepackage{enumitem}

% package for including images
\usepackage{graphicx}

% TAKEN FROM OVERLEAF DOCUMENTATION
% https://www.overleaf.com/learn/latex/Code_listing
\usepackage{listings}
\lstset{language=Python}
\usepackage{xcolor}
\definecolor{codegreen}{rgb}{0,0.6,0}
\definecolor{codegray}{rgb}{0.5,0.5,0.5}
\definecolor{codepurple}{rgb}{0.58,0,0.82}
\definecolor{backcolour}{rgb}{0.95,0.95,0.92}
\lstdefinestyle{mystyle}{
  backgroundcolor=\color{backcolour},
  commentstyle=\color{codegreen},
  keywordstyle=\color{magenta},
  numberstyle=\tiny\color{codegray},
  stringstyle=\color{codepurple},
  basicstyle=\ttfamily\footnotesize,
  breakatwhitespace=false,
  breaklines=true,
  captionpos=b,
  keepspaces=true,
  numbers=left,
  numbersep=5pt,
  showspaces=false,
  showstringspaces=false,
  showtabs=false,
  tabsize=2
}
\lstset{style=mystyle}

\newcommand{\vv}[1]{\mathbf{#1}}
\newcommand{\R}{\mathbb R}
\DeclareMathOperator{\vspan}{span}
\DeclareMathOperator{\cod}{cod}
\DeclareMathOperator{\ran}{ran}
\DeclareMathOperator{\col}{Col}
\DeclareMathOperator{\nul}{Nul}
\DeclareMathOperator{\rank}{rank}

% environment for solutions
\theoremstyle{remark}
\newtheorem*{solution}{Solution}

% capital letters for problem parts
\renewcommand{\theenumi}{\Alph{enumi}}

% no page numbers
\pagenumbering{gobble}

% UNCOMMENT IF YOU DON'T WANT PROBLEMS ON INDIVIDUAL PAGES
% \renewcommand{\pagebreak}{}

\title{
  Week 12 Discussion
}
\author{CAS CS 132: Geometric Algorithms}
\date{November 20, 2023}

\begin{document}
\maketitle

\noindent During discussion sections, we will go over three problems.
\begin{itemize}
\item The first will be a warm-up question, to help you verify your understanding of the material.
\item The second will be a solution to a problem on the assignment of the previous week.
\item The third will be a problem similar to one on the assignment of the following week.
\end{itemize}
The remainder of the time will be dedicated to open Q\&A.

\pagebreak
\section{Diagonalization by Hand}

Diagonalize the following matrix.
\begin{displaymath}
  A =
  \begin{bmatrix}
    -3 & -4 \\
    2 & 3
  \end{bmatrix}
\end{displaymath}

\medskip

\begin{solution}
  This implies the eigenvalues of this matrix are $1$ and $-1$.

  Next we have to find bases for the eigenspaces of each of these eigenvalues.
  First, we will find a basis for $\nul(A - I)$.
  Since
  \begin{displaymath}
    A - I =
    \begin{bmatrix}
      -4 & -4 \\
      2 & 2
    \end{bmatrix}
    \sim
    \begin{bmatrix}
      1 & 1 \\
      0 & 0
    \end{bmatrix}
  \end{displaymath}
  the solution set of the equation $(A - I)\vv x = \vv 0$ can be written as
  \begin{displaymath}
    x_2
    \begin{bmatrix}
      -1 \\ 1
    \end{bmatrix}
  \end{displaymath}
  Therefore, $[-1 \ 1]^T$ forms a basis of the eigenspace of $A$ for the eigenvalue $1$.

  For the eigenvalue $-1$, we can do the same:
  \begin{displaymath}
    A + I =
    \begin{bmatrix}
      -2 & -4 \\
      2 & 4
    \end{bmatrix}
    \sim
    \begin{bmatrix}
      1 & 2 \\
      0 & 0
    \end{bmatrix}
  \end{displaymath}
  which implies (by the same process as above) that $[-2 \ 1]^T$ forms a basis of the eigenspace of $A$ for $-1$.
  Therefore, we can take
  \begin{displaymath}
    P =
    \begin{bmatrix}
      -1 & -2 \\
      1 & 1
    \end{bmatrix}
    \qquad
    D =
    \begin{bmatrix}
      1 & 0 \\
      0 & -1
    \end{bmatrix}
  \end{displaymath}
  Finally, we have to find the inverse of $P$.
  One way to do this is to use the equation
  \begin{displaymath}
    \begin{bmatrix}
      a & b \\
      c & d
    \end{bmatrix}^{-1}
    =
    \frac{1}{ad - bc}
    \begin{bmatrix}
      d & -b \\
      -c & a
    \end{bmatrix}
  \end{displaymath}
  In this case,
  \begin{displaymath}
    P^{-1}=
    \frac{1}{(-1)(1) - (-2)(1)}
    \begin{bmatrix}
      1 & 2 \\
      -1 & -1
    \end{bmatrix}
    =
    \begin{bmatrix}
      1 & 2 \\
      -1 & -1
    \end{bmatrix}
  \end{displaymath}
\end{solution}

\pagebreak
\section{Determinant}
Consider the following reduction sequence
\begin{align*}
  R_1 &\gets R_1 + R_2 \\
  \mathsf{swap}&(R_2, R_3) \\
  R_3 &\gets R_3 + 5R_4 \\
  R_2 &\gets -3R_2 \\
  R_5 &\gets R_5 - 10R_3 \\
  R_5 &\gets R_5 / 11 \\
  \mathsf{swap}&(R_5, R_3) \\
  \mathsf{swap}&(R_1, R_2) \\
  R_4 &\gets R_4 + R_1 \\
  R_2 &\gets 5R_2 \\
  R_1 &\gets -R_1
\end{align*}
Suppose that $A \in \R^{5 \times 5}$ reduces to $U$ by this reduction sequence, where $U$ is in \textit{reduced} echelon form.
If $\rank A = 5$, what is $\det A$?
\medskip

\begin{solution}
\end{solution}


\pagebreak
\section{$2 \times 2$ Triangular Matrices and Diagonalization}

Consider an arbitrary $2 \times 2$ matrix
\begin{displaymath}
  A =
  \begin{bmatrix}
    a & b \\
    0 & d
  \end{bmatrix}
\end{displaymath}

\begin{enumerate}
\item
  Find an expression for the characteristic polynomial of $A$.
\item
  If $a \not = d$, then $A$ has 2 distinct eigenvalues, and so there must be an eigenbasis of $\R^2$ for $A$.
  If $a = d$, for what values of $b$ is there an eigenbasis of $\R^2$ for $A$?
  Justify your answer.
\end{enumerate}
\medskip

\begin{solution}
\end{solution}
\end{document}
