\documentclass{article}

% packages for math
\usepackage{amsthm}
\usepackage{amsmath}
\usepackage{amsfonts}

% package for including images
\usepackage{graphicx}

% TAKEN FROM OVERLEAF DOCUMENTATION
% https://www.overleaf.com/learn/latex/Code_listing
\usepackage{listings}
\lstset{language=Python}
\usepackage{xcolor}
\definecolor{codegreen}{rgb}{0,0.6,0}
\definecolor{codegray}{rgb}{0.5,0.5,0.5}
\definecolor{codepurple}{rgb}{0.58,0,0.82}
\definecolor{backcolour}{rgb}{0.95,0.95,0.92}
\lstdefinestyle{mystyle}{
    backgroundcolor=\color{backcolour},
    commentstyle=\color{codegreen},
    keywordstyle=\color{magenta},
    numberstyle=\tiny\color{codegray},
    stringstyle=\color{codepurple},
    basicstyle=\ttfamily\footnotesize,
    breakatwhitespace=false,
    breaklines=true,
    captionpos=b,
    keepspaces=true,
    numbers=left,
    numbersep=5pt,
    showspaces=false,
    showstringspaces=false,
    showtabs=false,
    tabsize=2
}
\lstset{style=mystyle}

% environment for solutions
\theoremstyle{remark}
\newtheorem*{solution}{Solution}

% capital letters for problem parts
\renewcommand{\theenumi}{\Alph{enumi}}

% no page numbers
\pagenumbering{gobble}

% UNCOMMENT IF YOU DON'T WANT PROBLEMS ON INDIVIDUAL PAGES
% \renewcommand{\pagebreak}{}

\title{Week 3 Discussion}
\author{CAS CS 132: Geometric Algorithms}
\date{September 18, 2023}

\begin{document}
\maketitle

\noindent During discussion sections, we will go over three problems.
\begin{itemize}
\item The first will be a warm-up question, to help you verify your understanding of the material.
\item The second will be a solution to a problem on the assignment of the previous week. \textbf{Note. Due to the extension, we won't be doing this today, but this document will be made available.}
\item The third will be a problem similar to one on the assignment of the following week.
\end{itemize}
The remainder of the time will be dedicated to open Q\&A.

\pagebreak
\section{Creating Inconsistent Systems (Warm-up)}

Find values $b_1$, $b_2$ and $b_3$ such that the system
\begin{align*}
  x_1 + x_2 + x_3 &= b_1 \\
  2x_1 + 2x_2 - x_3 &= b_2 \\
  x_1 + x_2 - 2x_3 &= b_3
\end{align*}
is inconsistent.

\begin{solution}
  The augmented matrix of this system is
  \begin{displaymath}
    \begin{bmatrix}
      1 & 1 & 1 & b_1 \\
      2 & 2 & -1 & b_2 \\
      1  & 1 & -2 & b_3
    \end{bmatrix}
  \end{displaymath}
  In order for this system to be inconsistent, it must be row equivalent to a matrix with a row of the form $[0 \ 0 \ \dots \ 0 \ a]$ where $a \not = 0$.
  So the first thing we need to do is row-reduce this matrix so that it has such a row.
  We then need to choose $b_1$, $b_2$, and $b_3$ so that row is inconsistent.
  In particular, it must be that its last entry is nonzero.

  By the row operation $R_2 \gets R_2 - R_1$ we get the matrix
  \begin{displaymath}
    \begin{bmatrix}
      1 & 1 & 1 & b_1 \\
      1 & 1 & -2 & b_2 - b_1 \\
      1 & 1 & -2 & b_3
    \end{bmatrix}
  \end{displaymath}
  Then by the row operation $R_3 \gets R_3 - R_2$, we get the matrix
  \begin{displaymath}
    \begin{bmatrix}
      1 & 1 & 1 & b_1 \\
      1 & 1 & -2 & b_2 - b_1 \\
      0 & 0 & 0 & b_3 - b_2 + b_1\\
    \end{bmatrix}
  \end{displaymath}
  So we need to chose $b_1$, $b_2$, and $b_3$ such that $b_3 - b_2 + b_1 \not = 0$.
  There are many choices, one simple choice is $b_1 = 0$, $b_2 = 0$ and $b_3 = 1$
\end{solution}

\pagebreak
\section{Pairwise Consistency (from HW 1)}

Given an example of a system of linear equations in 3 variables such that every pair of equations is consistent, but the system as a whole is inconsistent. That is, give values for $a$ through $l$ in the system
\begin{align*}
  ax + by + cz = d \\
  ex + fy + gz = h \\
  ix + jy + kz = l
\end{align*}
such that the system is inconsistent, but each pair of equations forms a consistent system. Present your solution to each part as the an augmented matrix, i.e., of the form
\begin{displaymath}
  \begin{bmatrix}
    a & b & c & d \\
    e & f & g & h \\
    i & j & k & l \\
  \end{bmatrix}
\end{displaymath}
(Hint. Make the third equation inconsistent with the sum of the first and second equation)
\begin{enumerate}
\item (5 pts) Achieve this with no more than 5 nonzero values for $a$ through $l$.
\item (5 pts) Achieve this with all nonzero values for $a$ through $l$.
\end{enumerate}

\begin{solution}
  Let's break down what this question is asking.
  To say that every pair of equations forms a consistent system, we require that each augmented matrix
  \begin{displaymath}
    \begin{bmatrix}
      a & b & c & d \\
      e & f & g & h
    \end{bmatrix}
    \qquad
    \begin{bmatrix}
      e & f & g & h \\
      i & j & k & l
    \end{bmatrix}
    \qquad
    \begin{bmatrix}
      a & b & c & d \\
      i & j & k & l
    \end{bmatrix}
  \end{displaymath}
  represents a consistent system.
  It then must also be the case that the augmented matrix
  \begin{displaymath}
    \begin{bmatrix}
      a & b & c & d \\
      e & f & g & h \\
      i & j & k & l
    \end{bmatrix}
  \end{displaymath}
  represents an inconsistent system.

  Now let's break down what this question means geometrically.
  If each pair of equations is consistent, then the planes they represent intersect.
  But if the entire is inconsistent, there is no point where all three planes intersect.

  This means that we cannot choose the planes so that two of them are parallel, since every plane needs to intersect every other plane at least once.
  What would such an orientation of planes look like?

  Finally, let's break down what this question means algebraically.
  Per the hint, we can choose a pair of consistent equations, and then find a third equation which is inconsistent with their sum.
  We also need to verify that this equation is consistent with the first two we wrote down.

  For part (a), since we're trying to minimize the number of nonzero values we use, we can start with the consistent system.
  \begin{displaymath}
    \begin{bmatrix}
      1 & 0 & 0 & 0 \\
      0 & 1 & 0 & 0
    \end{bmatrix}
  \end{displaymath}
  Geometrically, this is the $yz$-plane and the $xz$-plane (why?).
  If we take their sum, we get the equation: $x + y = 0$.
  The easiest way to write down an equation which is inconsistent with another is to write down an equation with the same coefficients but equal to a different value, e.g., $x + y = 1$ (it cannot possibly be the case that $x + y = 0$ and $x + y = 1$ simultaneously).
  This given us the entire system
  \begin{displaymath}
    \begin{bmatrix}
      1 & 0 & 0 & 0 \\
      0 & 1 & 0 & 0 \\
      1 & 1 & 0 & 1
    \end{bmatrix}
  \end{displaymath}
  By a cursory check, we see that each pair of systems is consistent.
  One tip: the only way that a pair of linear equations can be inconsistent is if they have the same coefficients up to scaling but are equal to a different value after scaling, e.g.,
  \begin{align*}
    x + 2y - z &= 1 \\
    2x + 4y - 2z &= 3 \\
  \end{align*}
\  This system has the augmented matrix
  \begin{displaymath}
    \begin{bmatrix}
      1 & 2 & -1 & 1 \\
      2 & 4 & -2 & 3
    \end{bmatrix}
  \end{displaymath}
  and the row-operation $R_2 \gets R_2 - 2R_1$ produces a matrix with an inconsistent row.

  This is not the case for any pair of rows in the solution above so each pair must be consistent.

  For the second part, we do the same thing, but we start with a pair of consistent equations without any zero values, e.g.,
  \begin{displaymath}
    \begin{bmatrix}
      1 & 1 & 1 & 1 \\
      1 & 2 & 1 & 1
    \end{bmatrix}
  \end{displaymath}
  The sum of the these two equations is: $2x + 3y + 2z = 2$.
  And an equation which is inconsistent with this is: $2x + 3y + 2z = 3$.
  This gives us the system
  \begin{displaymath}
    \begin{bmatrix}
      1 & 1 & 1 & 1\\
      1 & 2 & 1 & 1\\
      2 & 3 & 2 & 3
    \end{bmatrix}
  \end{displaymath}
  To verify, no pair of equations has the same coefficients up to scaling, so each pair is consistent.
  Also, the row-operations $R_3 = R_3 - R_2$ and $R_3 = R_3 - R_1$ produces a matrix with the row $[0 \ 0 \ \dots \ 0 \ 1]$.
\end{solution}

\pagebreak
\section{General Form Solutions (based on HW 2)}
Consider the following matrix in reduced echelon form.
\begin{displaymath}
  \begin{bmatrix}
    1 & 2 & 0 & 2 \\
    0 & 0 & 1 & 1
  \end{bmatrix}
\end{displaymath}
\begin{enumerate}
\item Write down the solution to this system in general form.
\item\label{b} Rewrite the solution in general form so that $x_1$ is free.
\item\label{c} Write down the solution from part \ref{b} as an augmented matrix (that is, your general form solution from part \ref{b} should have two linear equations, you should rearrange these equations and write them as an augmented matrix).
  Write down the row operation which converts the matrix from this part to the one above.
\end{enumerate}

\begin{solution}
  As we talked about in lecture, we have to identify the pivot positions of the matrix, as well as the basic and free variables.
  For each pivot column, we write down the equation for the variable of that column.
  For the non-pivot columns, we write that the variable is free.
  For this system we have
  \begin{align*}
    x_1 &= 2 - 2x_2 \\
    x_2 &\text{ is free} \\
    x_3 &= 1
  \end{align*}

  We can then rewrite this solution so $x_1$ is free.
  This means solving for $x_2$ in terms of $x_1$, which gives us $x_2 = 1 - (0.5)x_1$, and then rewriting the solution with $x_1$ as the free variable.
  \begin{align*}
    x_1 &\text{ is free} \\
    x_2 &= 1 - (0.5)x_1 \\
    x_3 &= 1
  \end{align*}
  Reading these two equations as a system of linear equations (note that we have to move the variables to the LHS of the equation) we get the augmented matrix
  \begin{displaymath}
    \begin{bmatrix}
      0.5 & 1 & 0 & 1 \\
      0 & 0 & 1 & 1 \\
    \end{bmatrix}
  \end{displaymath}
  This differs from our original matrix by the operation $R_1 \gets 2R_1$.
\end{solution}
\end{document}
