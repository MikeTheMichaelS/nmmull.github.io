\documentclass{article}

% packages for math
\usepackage{amsthm}
\usepackage{amsmath}
\usepackage{amssymb}
\usepackage{amsfonts}

\usepackage{enumitem}

% package for including images
\usepackage{graphicx}

% TAKEN FROM OVERLEAF DOCUMENTATION
% https://www.overleaf.com/learn/latex/Code_listing
\usepackage{listings}
\lstset{language=Python}
\usepackage{xcolor}
\definecolor{codegreen}{rgb}{0,0.6,0}
\definecolor{codegray}{rgb}{0.5,0.5,0.5}
\definecolor{codepurple}{rgb}{0.58,0,0.82}
\definecolor{backcolour}{rgb}{0.95,0.95,0.92}
\lstdefinestyle{mystyle}{
  backgroundcolor=\color{backcolour},
  commentstyle=\color{codegreen},
  keywordstyle=\color{magenta},
  numberstyle=\tiny\color{codegray},
  stringstyle=\color{codepurple},
  basicstyle=\ttfamily\footnotesize,
  breakatwhitespace=false,
  breaklines=true,
  captionpos=b,
  keepspaces=true,
  numbers=left,
  numbersep=5pt,
  showspaces=false,
  showstringspaces=false,
  showtabs=false,
  tabsize=2
}
\lstset{style=mystyle}

\newcommand{\vv}[1]{\mathbf{#1}}
\newcommand{\vspan}{\mathsf{span}}
\newcommand{\ran}{\mathsf{ran}}
\newcommand{\cod}{\mathsf{cod}}
\newcommand{\R}{\mathbb R}

% environment for solutions
\theoremstyle{remark}
\newtheorem*{solution}{Solution}

% capital letters for problem parts
\renewcommand{\theenumi}{\Alph{enumi}}

% no page numbers
\pagenumbering{gobble}

% UNCOMMENT IF YOU DON'T WANT PROBLEMS ON INDIVIDUAL PAGES
% \renewcommand{\pagebreak}{}

\title{
  Week 7 Discussion
}
\author{CAS CS 132: Geometric Algorithms}
\date{October 16, 2023}

\begin{document}
\maketitle

\noindent During discussion sections, we will go over three problems.
\begin{itemize}
\item The first will be a warm-up question, to help you verify your understanding of the material.
\item The second will be a solution to a problem on the assignment of the previous week.
\item The third will be a problem similar to one on the assignment of the following week.
\end{itemize}
The remainder of the time will be dedicated to open Q\&A.

\pagebreak
\section{Matrix Inverses (Warm Up)}
For each matrix $A$, determine if it is invertible. If it is, compute its inverse $A^{-1}$ and demonstrate that it is an inverse by computing $A^{-1}A$ and $AA^{-1}$.

\begin{enumerate}
\item
  \begin{displaymath}
    A =
    \begin{bmatrix}
      2 & 3 & -1 \\
      0 & -1 & 1 \\
      2 & 1 & 1
    \end{bmatrix}
  \end{displaymath}
\item
  \begin{displaymath}
    A =
    \begin{bmatrix}
      1 & 2 & 0 \\
      0 & -1 & 1 \\
      2 & 2 & 0
    \end{bmatrix}
  \end{displaymath}
\end{enumerate}

\begin{solution}
\end{solution}

\pagebreak
\section{Span and Linear Independence (Midterm)}
Consider the following vectors in $\mathbb R^4$.
\begin{displaymath}
  \mathbf v_1 =
  \begin{bmatrix}
    -3 \\
    4 \\
    3 \\
    7
  \end{bmatrix}
  \mathbf v_2 =
  \begin{bmatrix}
    -1 \\
    0 \\
    0 \\
    2
  \end{bmatrix}
  \mathbf v_3 =
  \begin{bmatrix}
    0 \\
    -2 \\
    -1 \\
    -1
  \end{bmatrix}
  \mathbf v_4 =
  \begin{bmatrix}
    1 \\
    1 \\
    1 \\
    -2
  \end{bmatrix}
\end{displaymath}
\begin{enumerate}
\item Determine if $\vv v_1$ is in $\vspan\{\vv v_2, \vv v_3, \vv v_4\}$.
  Justify your answer.
  In particular, if $\vv v_1$ is in $\vspan\{\vv v_2, \vv v_3, \vv v_4\}$, then write $\vv v_1$ as a linear combination of $\vv v_2$, $\vv v_3$, and $\vv v_4$.
\item Determine if the vectors $\vv v_2$, $\vv v_3$, and $\vv v_4$ are linearly independent.
  Justify your answer.
  In particular, if they are linearly dependent, then write a dependence relation for them (that is, write the zero vector $\vv 0$ as a linear combination of the vectors $\vv v_2$, $\vv v_3$, and $\vv v_4$).
\item Determine if the vectors $\vv v_1$, $\vv v_2$, and $\vv v_3$ are linearly independent.
  Justify your answer.
  In particular, if they are linearly dependent, then write a dependence relation for them.
\end{enumerate}

\begin{solution}
\end{solution}


\pagebreak
\section{3D Rotation Matrices}

The matrices used for rotating counter-clockwise by $\theta$ around the $x$-, $y$-, and $z$-axes are
\begin{displaymath}
  R_{x, \theta} =
  \begin{bmatrix}
    1 & 0 & 0 \\
    0 & \cos\theta & -\sin\theta \\
    0 & \sin\theta & \cos\theta
  \end{bmatrix}
  \quad
  R_{y, \theta} =
  \begin{bmatrix}
    \cos\theta & 0 & \sin\theta \\
    0 & 1 & 0 \\
    -\sin\theta & 0 & \cos\theta
  \end{bmatrix}
  \quad
  R_{z, \theta} =
  \begin{bmatrix}
    \cos\theta & -\sin \theta & 0 \\
    \sin\theta & \cos\theta & 0 \\
    0 & 0 & 1
  \end{bmatrix}
\end{displaymath}
Note that difference in sign for $R_{y, \theta}$.
This is due to a convention called the \textit{right-hand rule} which says that the orientation of the positive directions of each axis is given by thinking of the index finger, middle finger, and thumb of our right hand as the $x$-, $y$-, and $z-$axes (and that when we talk about counter-clockwise around an axis, we mean counter-clockwise when the positive axis is pointing towards us).

\medskip

\begin{enumerate}
\item Compute $\vv v = R_{z, 45^\circ} [1 \ (-1) \ \sqrt{2}]^T$. Recall that $\cos45^\circ = \sin45^\circ = \frac{\sqrt{2}}{2}$.
\item Computer $R_{y, 45^\circ}\vv v$, where $\vv v$ is the vector from the previous part.
\end{enumerate}
Try to draw, to the best of your ability, what is happening to this vector after each transformation.
\medskip

\begin{solution}
\end{solution}

\end{document}
