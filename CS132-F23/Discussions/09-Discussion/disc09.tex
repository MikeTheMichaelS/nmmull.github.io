\documentclass{article}

% packages for math
\usepackage{amsthm}
\usepackage{amsmath}
\usepackage{amssymb}
\usepackage{amsfonts}

\usepackage{enumitem}

% package for including images
\usepackage{graphicx}

% TAKEN FROM OVERLEAF DOCUMENTATION
% https://www.overleaf.com/learn/latex/Code_listing
\usepackage{listings}
\lstset{language=Python}
\usepackage{xcolor}
\definecolor{codegreen}{rgb}{0,0.6,0}
\definecolor{codegray}{rgb}{0.5,0.5,0.5}
\definecolor{codepurple}{rgb}{0.58,0,0.82}
\definecolor{backcolour}{rgb}{0.95,0.95,0.92}
\lstdefinestyle{mystyle}{
  backgroundcolor=\color{backcolour},
  commentstyle=\color{codegreen},
  keywordstyle=\color{magenta},
  numberstyle=\tiny\color{codegray},
  stringstyle=\color{codepurple},
  basicstyle=\ttfamily\footnotesize,
  breakatwhitespace=false,
  breaklines=true,
  captionpos=b,
  keepspaces=true,
  numbers=left,
  numbersep=5pt,
  showspaces=false,
  showstringspaces=false,
  showtabs=false,
  tabsize=2
}
\lstset{style=mystyle}

\newcommand{\vv}[1]{\mathbf{#1}}
\newcommand{\vspan}{\mathsf{span}}
\newcommand{\ran}{\mathsf{ran}}
\newcommand{\cod}{\mathsf{cod}}
\newcommand{\R}{\mathbb R}

% environment for solutions
\theoremstyle{remark}
\newtheorem*{solution}{Solution}

% capital letters for problem parts
\renewcommand{\theenumi}{\Alph{enumi}}

% no page numbers
\pagenumbering{gobble}

% UNCOMMENT IF YOU DON'T WANT PROBLEMS ON INDIVIDUAL PAGES
% \renewcommand{\pagebreak}{}

\title{
  Week 9 Discussion
}
\author{CAS CS 132: Geometric Algorithms}
\date{October 30, 2023}

\begin{document}
\maketitle

\noindent During discussion sections, we will go over three problems.
\begin{itemize}
\item The first will be a warm-up question, to help you verify your understanding of the material.
\item The second will be a solution to a problem on the assignment of the previous week.
\item The third will be a problem similar to one on the assignment of the following week.
\end{itemize}
The remainder of the time will be dedicated to open Q\&A.

\pagebreak
\section{LU Factorization by Hand (Warm Up)}

Compute the LU factorization of the following matrix.
Then verify that your factorization is correct by carrying out the matrix multiplication $LU$.

\begin{displaymath}
  \begin{bmatrix}
    1 & 2 & -3 \\
    -1 & -1 & 4 \\
    2 & 3 & -6
  \end{bmatrix}
\end{displaymath}

\medskip

\begin{solution}
\end{solution}

\pagebreak
\section{Matrix Algebra}

\begin{enumerate}
\item Suppose $A$ and $B$ are invertible matrices and $AB^TXA^{-1}B = I$. Solve for $X$ in terms of $A$ and $B$.
\item Show that $A + A^T$ is symmetric for any square matrix $A$. That is, show that $(A + A^T)^T = A + A^T$.
\item Show that if $A$ and $B$ are symmetric and $AB = BA$ then $AB$ is symmetric.
\end{enumerate}

\medskip

\begin{solution}
\end{solution}

\pagebreak
\section{NumPy and SciPy}

\begin{enumerate}
\item
  Construct a random $10 \times 10$ matrix and a random vector in $\R^{10}$ using
  \begin{lstlisting}
import numpy
import scipy

a = numpy.random.rand(10, 10)
b = numpy.random.rand(10)\end{lstlisting}

  Then construct its inverse and its LU factoration using \texttt{numpy.linalg.inv} and \texttt{scipy.linalg.lu\_factor}, respectively.
  Finally, solve equation \texttt{ax = b} in three ways:
  \begin{lstlisting}
s1 = numpy.linalg.solve(a, b)
s2 = a_inv @ b
s3 = scipy.lu_solve(lu) # where lu is the result of lu_factor\end{lstlisting}
  Verify that these three solutions are the same.
  Note, this process is not guaranteed to succeed since not all square matrices are invertible, but it is incredibly unlikely that a random $10 \times 10$ matrix is not invertible.
\item
  Construct a random integer matrix with entries between 1 and 10 as follows:
  \begin{lstlisting}
a = numpy.random.randint(1, 11, (10, 10))\end{lstlisting}
  Write a function called \texttt{norm\_col} which divides a NumPy vector by the sum of its entries (you don't need to consider the case in which the input is the zero vector).
  Then read about the function \texttt{numpy.apply\_along\_axis} in the NumPy documentation.
  Check the value of
  \begin{lstlisting}
numpy.apply_along_axis(norm_col, 0, a)\end{lstlisting}
\end{enumerate}
\end{document}
