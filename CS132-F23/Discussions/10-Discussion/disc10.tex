\documentclass{article}

% packages for math
\usepackage{amsthm}
\usepackage{amsmath}
\usepackage{amssymb}
\usepackage{amsfonts}

\usepackage{enumitem}

% package for including images
\usepackage{graphicx}

% TAKEN FROM OVERLEAF DOCUMENTATION
% https://www.overleaf.com/learn/latex/Code_listing
\usepackage{listings}
\lstset{language=Python}
\usepackage{xcolor}
\definecolor{codegreen}{rgb}{0,0.6,0}
\definecolor{codegray}{rgb}{0.5,0.5,0.5}
\definecolor{codepurple}{rgb}{0.58,0,0.82}
\definecolor{backcolour}{rgb}{0.95,0.95,0.92}
\lstdefinestyle{mystyle}{
  backgroundcolor=\color{backcolour},
  commentstyle=\color{codegreen},
  keywordstyle=\color{magenta},
  numberstyle=\tiny\color{codegray},
  stringstyle=\color{codepurple},
  basicstyle=\ttfamily\footnotesize,
  breakatwhitespace=false,
  breaklines=true,
  captionpos=b,
  keepspaces=true,
  numbers=left,
  numbersep=5pt,
  showspaces=false,
  showstringspaces=false,
  showtabs=false,
  tabsize=2
}
\lstset{style=mystyle}

\newcommand{\vv}[1]{\mathbf{#1}}
\newcommand{\R}{\mathbb R}
\DeclareMathOperator{\vspan}{span}
\DeclareMathOperator{\cod}{cod}
\DeclareMathOperator{\ran}{ran}
\DeclareMathOperator{\col}{Col}
\DeclareMathOperator{\nul}{Nul}
\DeclareMathOperator{\rank}{rank}

% environment for solutions
\theoremstyle{remark}
\newtheorem*{solution}{Solution}

% capital letters for problem parts
\renewcommand{\theenumi}{\Alph{enumi}}

% no page numbers
\pagenumbering{gobble}

% UNCOMMENT IF YOU DON'T WANT PROBLEMS ON INDIVIDUAL PAGES
% \renewcommand{\pagebreak}{}

\title{
  Week 10 Discussion
}
\author{CAS CS 132: Geometric Algorithms}
\date{November 6, 2023}

\begin{document}
\maketitle

\noindent During discussion sections, we will go over three problems.
\begin{itemize}
\item The first will be a warm-up question, to help you verify your understanding of the material.
\item The second will be a solution to a problem on the assignment of the previous week.
\item The third will be a problem similar to one on the assignment of the following week.
\end{itemize}
The remainder of the time will be dedicated to open Q\&A.

\pagebreak
\section{Column Space and Null Space (Warm Up)}

Consider the following matrix

\begin{displaymath}
  A =
  \begin{bmatrix}
    1 & -2 & 0 \\
    0 & 1 & -1 \\
    -3 & 1 & 5
  \end{bmatrix}
\end{displaymath}

\begin{enumerate}
\item Find the reduced echelon form of $A$.
\item Write down a general form solution for the equation $A\vv x = \vv 0$.
\item Find a basis for $\col A$.
\item Find a basis for $\nul A$.
\item Find a linear equation whose solution set (i.e., those points in the plane represented by the linear equation) is $\col A$.
\end{enumerate}

\medskip

\begin{solution}
\end{solution}

\pagebreak
\section{Order of Matrix Multiplcation}

The matrices for the 2D transformations (on homogeneous coordinates) of translation and counterclockwise rotation about the origin are
\begin{displaymath}
  T_{x,y} =
  \begin{bmatrix}
    1 & 0 & x \\
    0 & 1 & y \\
    0 & 0 & 1
  \end{bmatrix}
  \qquad
  R_{\theta} =
  \begin{bmatrix}
    \cos \theta & -\sin \theta & 0 \\
    \sin \theta & \cos \theta & 0 \\
    0 & 0 & 1
  \end{bmatrix}
\end{displaymath}

\begin{enumerate}
\item Show that $T_{1, 1} R_{\pi / 4} \not = R_{\pi / 4} T_{1, 1}$.
  Recall that $\pi / 4$ in radians is $45^\circ$ and $\cos(\pi / 4) = \sin(\pi / 4) = \sqrt{2} / 2$.
\item If we want to do rotation \textit{and then} translation, which of the two matrices in the previous part do we want to use?
\item Draw the effects $T_{1, 1} R_{
  \pi / 4}$ and $R_{\pi / 4} T_{1, 1}$ on the unit square on two separate graphs.
\end{enumerate}

\medskip

\begin{solution}
\end{solution}

\pagebreak
\section{Intersections and Unions of Subspaces}

Let $H_1$ and $H_2$ be arbitrary subspaces of $\R^n$.

\begin{enumerate}
\item The \textit{intersection} of $H_1$ and $H_2$, written $H_1 \cap H_2$, is the set of vectors which appear in \textit{both} $H_1$ and $H_2$:
  \begin{displaymath}
    H_1 \cap H_2 = \{ \vv v : \vv v \in H_1 \text{ and } \vv v \in H_2 \}
  \end{displaymath}
  Show that $H_1 \cap H_2$ is a subspace. (In other words, the intersection of two spans is a span).
\item The \textit{union} of $H_1$ and $H_2$, written $H_1 \cup H_2$, is the set of vectors which appear in \textit{at least one of} $H_1$ and $H_2$:
  \begin{displaymath}
    H_1 \cup H_2 = \{\vv v : \vv v \in H_1 \text{ or } \vv v \in H_2 \}
  \end{displaymath}
  Give an explicit example of two subspaces $\R^2$ whose union is \textit{not} a subspace of $\mathbb R^2$.
  \textit{Hint.} Pick pretty much any two vectors $\vv u$ and $\vv v$ in $\R^2$ and take the two subspaces to be $\vspan\{\vv u\}$ and $\vspan \{ \vv v \}$.
\end{enumerate}

\medskip

\begin{solution}
\end{solution}

\end{document}
