\documentclass{article}

% packages for math
\usepackage{amsthm}
\usepackage{amsmath}
\usepackage{amssymb}
\usepackage{amsfonts}

\usepackage{enumitem}

% package for including images
\usepackage{graphicx}

% TAKEN FROM OVERLEAF DOCUMENTATION
% https://www.overleaf.com/learn/latex/Code_listing
\usepackage{listings}
\lstset{language=Python}
\usepackage{xcolor}
\definecolor{codegreen}{rgb}{0,0.6,0}
\definecolor{codegray}{rgb}{0.5,0.5,0.5}
\definecolor{codepurple}{rgb}{0.58,0,0.82}
\definecolor{backcolour}{rgb}{0.95,0.95,0.92}
\lstdefinestyle{mystyle}{
  backgroundcolor=\color{backcolour},
  commentstyle=\color{codegreen},
  keywordstyle=\color{magenta},
  numberstyle=\tiny\color{codegray},
  stringstyle=\color{codepurple},
  basicstyle=\ttfamily\footnotesize,
  breakatwhitespace=false,
  breaklines=true,
  captionpos=b,
  keepspaces=true,
  numbers=left,
  numbersep=5pt,
  showspaces=false,
  showstringspaces=false,
  showtabs=false,
  tabsize=2
}
\lstset{style=mystyle}

\newcommand{\vv}[1]{\mathbf{#1}}
\newcommand{\R}{\mathbb R}
\DeclareMathOperator{\vspan}{span}
\DeclareMathOperator{\cod}{cod}
\DeclareMathOperator{\ran}{ran}
\DeclareMathOperator{\col}{Col}
\DeclareMathOperator{\nul}{Nul}
\DeclareMathOperator{\rank}{rank}

% environment for solutions
\theoremstyle{remark}
\newtheorem*{solution}{Solution}

% capital letters for problem parts
\renewcommand{\theenumi}{\Alph{enumi}}

% no page numbers
\pagenumbering{gobble}

% UNCOMMENT IF YOU DON'T WANT PROBLEMS ON INDIVIDUAL PAGES
% \renewcommand{\pagebreak}{}

\title{
  Week 14 Discussion
}
\author{CAS CS 132: Geometric Algorithms}
\date{December 4, 2023}

\begin{document}
\maketitle

\noindent During discussion sections, we will go over three problems.
\begin{itemize}
\item The first will be a warm-up question, to help you verify your understanding of the material.
\item The second will be a solution to a problem on the assignment of the previous week.
\item The third will be a problem similar to one on the assignment of the following week.
\end{itemize}
The remainder of the time will be dedicated to open Q\&A.

\pagebreak
\section{Basis of the column space (Warm up)}

Consider the following matrices. Note that $A'$ is an echelon from of $A$.
\begin{displaymath}
  A =
  \begin{bmatrix}
    1 & 1 & 2 & 0 & 2 \\
    3 & 4 & 9 & -2 & 5 \\
    -2 & -3 & -7 & 2 & -2 \\
    2 & 2 & 4 & 0 & 5
  \end{bmatrix}
  \qquad
  A' =
  \begin{bmatrix}
    1 & 0 & -1 & 2 & 0 \\
    0 & 1 & 3 & -2 & 0 \\
    0 & 0 & 0 & 0 & 1 \\
    0 & 0 & 0 & 0 & 0
  \end{bmatrix}
\end{displaymath}

\begin{enumerate}
\item Use the echelon form above to find a basis of $\col A$ made up of columns of $A$.
\item Write down a NumPy expression in terms of \texttt{A} (a 2D NumPy array representing the matrix $A$ above) for the matrix whose columns are the basis vectors you found in the previous part.
\item Let $A_i$ be the matrix whose columns are the first $i$ columns of $A$. For example,
  \begin{displaymath}
    A_3 =
    \begin{bmatrix}
      1 & 1 & 2 \\
      3 & 4 & 9 \\
      -2 & -3 & -7 \\
      2 & 2 & 4
    \end{bmatrix}
  \end{displaymath}
  Find $\rank(A_i)$ for each $i$ using the echelon form above.
\item Write down a NumPy expression for $\rank(A_i)$ in terms of \texttt{A} and \texttt{i} and the NumPy function \texttt{numpy.linalg.matrix\_rank}, which returns the rank of its argument.
\item
  Let $B$ be an arbitrary $m \times 5$ matrix and let $B_i$ be the matrix whose columns are the first $i$ columns of $B$.
  Further suppose that $\rank(B_1) = 1$, $\rank(B_2) = 1$, $\rank(B_3) = 2$, $\rank(B_4) = 3$, and $\rank(B_5) = 3$.
  Which columns of $B$ form a basis of $\col B$?
\item Use the previous parts to describe in an informal procedure you can use to find a basis for the column space of a small matrix using Python.
\end{enumerate}

\medskip

\begin{solution}
\end{solution}

\pagebreak
\section{Boundary Reflection without a Matrix}

Suppose that $A$ is a $n \times n$ matrix and $\vv z$ is a vector in $\R^n$ whose $i$th component is $1$ if the $i$th column of $A$ is $\vv 0$, and $0$ otherwise, e.g.,
\begin{displaymath}
  A =
  \begin{bmatrix}
    0 & 1 & 0 \\
    0 & -3 & 0 \\
    0 & 2 & 0
  \end{bmatrix}
  \qquad
  \vv z =
  \begin{bmatrix}
    1 \\ 0 \\ 1
  \end{bmatrix}
\end{displaymath}
For an arbitrary vector $\vv v$ in $\R^n$, write down an expression in terms of $A$, $\vv z$ and $\vv v$ for the vector
\begin{displaymath}
  A' \vv v
\end{displaymath}
where $A'$ is the same as $A$, but every all-zeros column of $A$ is replaced with the vector $c\vv 1$ for some scalar $c$, e.g., as it pertains to the example above,
\begin{displaymath}
  A' =
  \begin{bmatrix}
    c & 1 & c \\
    c & -3 & c \\
    c & 2 & c
  \end{bmatrix}
\end{displaymath}
Furthermore, write down a NumPy expression which computes this \textit{without} using the function \texttt{numpy.ones}.

\begin{solution}
\end{solution}

\pagebreak
\section{Multiple Least Squares Solutions}

\begin{displaymath}
  A =
  \begin{bmatrix}
    1 & 1 & 0 & 0 \\
    1 & 1 & 0 & 0 \\
    1 & 0 & 1 & 0 \\
    1 & 0 & 1 & 0 \\
    1 & 0 & 0 & 1 \\
    1 & 0 & 0 & 1 \\
  \end{bmatrix}
  \qquad
  \vv b =
  \begin{bmatrix}
    -3 \\ 1 \\ 0 \\ 2 \\ 5 \\ 1
  \end{bmatrix}
\end{displaymath}
\begin{enumerate}
\item Find the orthogonal projection $\hat{\vv b}$ onto $\col A$. (\textit{Hint.} Note that the columns of $A$ are linearly dependent. It will be easier to do the computation if you take the last three columns of $A$ to find the projection.)
\item Find a general form solution for the homogeneous equation $A^TA\vv x = \vv 0$. Then write this general form solution as a linear combination of vectors with free variables as weights.
\item Find the normal equations for the system $A\vv x = \vv b$.
\item Using the normal equations find a general form solution for the set of least squares solutions of $A\vv x = \vv b$. Then write this general form solution as a linear combination of vectors with free variables \textbf{and the scalar 1} as weights.
\end{enumerate}
\medskip

\begin{solution}
\end{solution}
\end{document}
