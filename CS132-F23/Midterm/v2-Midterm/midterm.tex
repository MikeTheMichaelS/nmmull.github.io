\documentclass{article}

% packages for math
\usepackage{amsthm}
\usepackage{amsmath}
\usepackage{amssymb}
\usepackage{amsfonts}

% package for including images
\usepackage{graphicx}

% TAKEN FROM OVERLEAF DOCUMENTATION
% https://www.overleaf.com/learn/latex/Code_listing
\usepackage{listings}
\lstset{language=Python}
\usepackage{xcolor}
\definecolor{codegreen}{rgb}{0,0.6,0}
\definecolor{codegray}{rgb}{0.5,0.5,0.5}
\definecolor{codepurple}{rgb}{0.58,0,0.82}
\definecolor{backcolour}{rgb}{0.95,0.95,0.92}
\lstdefinestyle{mystyle}{
  backgroundcolor=\color{backcolour},
  commentstyle=\color{codegreen},
  keywordstyle=\color{magenta},
  numberstyle=\tiny\color{codegray},
  stringstyle=\color{codepurple},
  basicstyle=\ttfamily\footnotesize,
  breakatwhitespace=false,
  breaklines=true,
  captionpos=b,
  keepspaces=true,
  numbers=left,
  numbersep=5pt,
  showspaces=false,
  showstringspaces=false,
  showtabs=false,
  tabsize=2
}
\lstset{style=mystyle}

% environment for solutions
\theoremstyle{remark}
\newtheorem*{solution}{Solution}

% capital letters for problem parts
\renewcommand{\theenumi}{\Alph{enumi}}

% no page numbers
\pagenumbering{gobble}

\newcommand{\vv}[1]{\mathbf{#1}}
\newcommand{\vspan}{\mathsf{span}}
\newcommand{\ran}{\mathsf{ran}}
\newcommand{\cod}{\mathsf{cod}}
\newcommand{\R}{\mathbb R}

% UNCOMMENT IF YOU DON'T WANT PROBLEMS ON INDIVIDUAL PAGES
% \renewcommand{\pagebreak}{}

\title{
  Midterm
}
\author{CAS CS 132: Geometric Algorithms}
\date{October 12, 2023}

\begin{document}
\maketitle

\noindent Name:

\bigskip

\noindent BUID:

\bigskip

\noindent Location:

\bigskip

\begin{itemize}
\item You will have approximately 70 minutes to complete this exam.
\item Make sure to read every question, some are easier than others.
\item Please write your name and BUID \textbf{on every page}.
\item Your work will only be seens on the \textbf{front facing pages}.
  You may write on the backs of pages if you are using a pencil, but you will not be evaluated on what is written there.
\end{itemize}

\pagebreak
\section{Span and Linear Independence}
Consider the following vectors in $\mathbb R^4$.
\begin{displaymath}
  \mathbf v_1 =
  \begin{bmatrix}
    -1 \\
    -6 \\
    0 \\
    -7
  \end{bmatrix}
  \mathbf v_2 =
  \begin{bmatrix}
    1 \\
    -2 \\
    -3 \\
    -1
  \end{bmatrix}
  \mathbf v_3 =
  \begin{bmatrix}
    1 \\
    -1 \\
    -2 \\
    0
  \end{bmatrix}
  \mathbf v_4 =
  \begin{bmatrix}
    0 \\
    -1 \\
    0 \\
    -1
  \end{bmatrix}
\end{displaymath}
\begin{enumerate}
\item (6 points) Determine if $\vv v_1$ is in $\vspan\{\vv v_2, \vv v_3, \vv v_4\}$.
  Justify your answer.
  In particular, if $\vv v_1$ is in $\vspan\{\vv v_2, \vv v_3, \vv v_4\}$, then write $\vv v_1$ as a linear combination of $\vv v_2$, $\vv v_3$, and $\vv v_4$.
\item (6 points) Determine if the vectors $\vv v_2$, $\vv v_3$, and $\vv v_4$ are linearly independent.
  Justify your answer.
  In particular, if they are linearly dependent, then write a dependence relation for them (that is, write the zero vector $\vv 0$ as a linear combination of the vectors $\vv v_2$, $\vv v_3$, and $\vv v_4$).
\item (6 points) Determine if the vectors $\vv v_1$, $\vv v_2$, and $\vv v_3$ are linearly independent.
  Justify your answer.
  In particular, if they are linearly dependent, then write a dependence relation for them.
\end{enumerate}

\begin{solution}
\end{solution}

\pagebreak
\noindent\textit{Extra Page.}

\pagebreak
\section{True/False Questions}

For each of the following statements, determine if they are true or false.
You do not have to show your work or justify your answer. You just have to write ``true'' or ``false.''

\begin{enumerate}
\item (2 points) For any $20 \times 24$ matrix $A$, the equation $A\vv x = \vv 0$ has a nontrivial solution.
\item (2 points) For any $m \times n$ matrix $A$, if $m > n$ then the columns of $A$ must be linearly independent.
\item (2 points) Every matrix has a unique echelon form.
\item (2 points) For any nonzero real values $a$ and $b$, the matrix
  $
  \begin{bmatrix}
    a & 2a \\
    b & 3b
  \end{bmatrix}
  $
  has a pivot in every column and every row.
\item (2 points) If $A$ is the augmented matrix of an inconsistent system, then $A$ has a pivot in its last (rightmost) column.
\item (2 points) To show that a transformation $T: \R^m \to \R^n$ is linear, it is enough to show that $T(\vv x + \vv y) = T(\vv x) + T(\vv y)$ for any two vectors $\vv x$ and $\vv y$ in $\R^m$.
\item (2 points) For any vectors $\vv v_1$, $\vv v_2$, and $\vv v_3$ in $\R^n$, if $\{\vv v_1, \vv v_2, \vv v_3\}$ is linearly independent, then so is $\{\vv v_1, \vv v_2\}$.
\item (2 points) Any set of distinct standard basis vectors in $\mathbb R^n$ is linearly independent.
\end{enumerate}

\begin{solution}
  \hfill
  \begin{enumerate}
  \item
  \item
  \item
  \item
  \item
  \item
  \item
  \item
  \end{enumerate}
\end{solution}

\pagebreak
\noindent\textit{Extra Page.}

\pagebreak
\section{Inner Products and Matrix Equations}

Consider the following matrix and vector.\footnote{Credit to Vishesh Jain for suggesting a version of this problem.}
\begin{displaymath}
  A =
  \begin{bmatrix}
    1 & -1 & 1 \\
    -1 & 2 & -1 \\
    1 & -2 & 1
  \end{bmatrix}
  \qquad
  \vv b =
  \begin{bmatrix}
    1 \\ 0 \\ 0
  \end{bmatrix}
\end{displaymath}
\begin{enumerate}
\item (3 points) Compute the following matrix-vector multiplications.
  \begin{displaymath}
    \begin{bmatrix}
      1 & -1 & 1
    \end{bmatrix}
    \left(
    A
    \begin{bmatrix}
      1 \\ -1 \\ 1
    \end{bmatrix}
    \right)
  \end{displaymath}
\item (6 points) Write down a general form solution for the solution set of the  matrix equation $A \vv x = \vv b$.
\item (6 points) Use your solution to the previous part to find a \textbf{nonzero} vector $\begin{bmatrix} v_1 \\ v_2 \\ v_3 \end{bmatrix}$ such that
  \begin{displaymath}
    \begin{bmatrix}
      v_1 & v_2 & v_3
    \end{bmatrix}
    \left(A
    \begin{bmatrix}
      v_1 \\ v_2 \\ v_3
    \end{bmatrix}\right)
    = 0
  \end{displaymath}
\end{enumerate}

\begin{solution}
\end{solution}
\vfill

\pagebreak
\noindent\textit{Extra Page.}

\pagebreak
\section{Linear Transformations}

(5 points) Consider the following linear transformation $T$.
\begin{displaymath}
  \begin{bmatrix}
    x_1 \\ x_2 \\ x_3
  \end{bmatrix}
  \mapsto
  \begin{bmatrix}
    x_1 + 2x_2 - x_3 \\
    x_3 \\
    2x_3
  \end{bmatrix}
\end{displaymath}
Find a set of linearly independent vectors which span the range of $T$. \textit{Hint.} First find the matrix implementing $T$.

\begin{solution}
\end{solution}

\pagebreak
\section{Matrix Equations}
Consider the following matrices.
\begin{displaymath}
  B =
  \begin{bmatrix}
    1 & 1 & 0 & 0 & 0 & 0 \\
    0 & 1 & 1 & 0 & 0 & 0 \\
    0 & 0 & 1 & 1 & 0 & 0 \\
    0 & 0 & 0 & 1 & 1 & 0 \\
    0 & 0 & 0 & 0 & 1 & 1 \\
    0 & 0 & 0 & 0 & 0 & 1
  \end{bmatrix}
  \quad
  C =
  \begin{bmatrix}
    1 & 1 & 0 & 0 & 0 & 0 \\
    1 & 1 & 1 & 0 & 0 & 0 \\
    0 & 1 & 1 & 1 & 0 & 0 \\
    0 & 0 & 1 & 1 & 1 & 0 \\
    0 & 0 & 0 & 1 & 1 & 1 \\
    0 & 0 & 0 & 0 & 1 & 1
  \end{bmatrix}
  \quad
  \vv e_6 =
  \begin{bmatrix}
    0 \\ 0 \\ 0 \\ 0 \\ 0 \\ 1
  \end{bmatrix}
\end{displaymath}

\begin{enumerate}
\item (3 points) Explain why the columns of $B$ span $\R^6$.
\item (7 points) Find a solution to the equation $B \vv x = \vv e_6$.
\item \textbf{(3 points, Extra Credit)} Find a solution to the equation $C \vv x = \vv e_6$.
\end{enumerate}

\begin{solution}
\end{solution}

\pagebreak
\noindent\textit{Extra Page.}
\end{document}
